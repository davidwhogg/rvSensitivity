% This document is part of the rvSensitivity project.
% Copyright 2020 the authors. All rights reserved.

% to-do
% -----
% - draft
% - send to friendlies
% - post to arXiv
% - profit

% style notes
% -----------
% - use \, for multiplication
% - put punctuation after equations as necessary, with "\hquad,". Use \hquad
% - put term name definitions in \textsl{}, *not* quotations
% - don't say "i.e.," say "that is,". Don't say "e.g.," say "for example,".
% - ...[your pet peeve here]...

\documentclass[10pt, letterpaper]{article}

\usepackage{amsmath, bm, mathrsfs, amssymb}
\usepackage[dvipsnames]{xcolor}
\usepackage[hidelinks,
            colorlinks=true,
            linkcolor=NavyBlue,
            citecolor=darkgray,
            urlcolor=NavyBlue]{hyperref}
\usepackage{pifont}
\usepackage{graphicx}
\usepackage{doi}

\usepackage{natbib}
\bibliographystyle{hogg_abbrvnat}
\setcitestyle{round,citesep={,},aysep={}}

% text stuhh
\newcommand{\documentname}{\textsl{Note}}
\newcommand{\sectionname}{Section}
\newcommand{\equationname}{equation}
\newcommand{\notename}{Note}
\renewcommand{\tablename}{Table}
\newcommand{\acronym}[1]{{\small{#1}}}
\newcommand{\code}[1]{\texttt{#1}}
\newcommand{\foreign}[1]{\textsl{#1}}
\newcommand{\etal}{\foreign{et~al.}}

% math stuhh
\newcommand{\hquad}{~~}
\newcommand{\given}{\,|\,}
\newcommand{\dd}{\mathrm{d}}
\newcommand{\T}{^{\!\mathsf{T}\!}}
\newcommand{\inv}{^{-1}}
\newcommand{\scalar}[1]{#1}
\renewcommand{\vector}[1]{\boldsymbol{#1}}
\newcommand{\tensor}[1]{\mathbf{#1}}
\renewcommand{\matrix}[1]{\mathsf{#1}}
\newcommand{\normal}{\mathcal{N}\!\,}
\newcommand{\uniform}{\mathcal{U}\!\,}
\newcommand{\unit}[1]{\mathrm{#1}}
\newcommand{\BJD}{\unit{BJD}}
\newcommand{\kms}{\unit{km}\,\unit{s}^{-1}}

% Journal keys - kill me now
\newcommand{\aj}{Astron.\,J.}
\newcommand{\apj}{Astrophys.\,J.}
\newcommand{\apjs}{Astrophys.\,J.\,Supp.\,Ser.}

% page layout stuff
\setlength{\topmargin}{-0.2in}
\setlength{\headheight}{0in}
\setlength{\headsep}{0in}
\setlength{\textheight}{9.35in}
\setlength{\parindent}{\baselineskip}
\setlength{\oddsidemargin}{-0.2in}
\setlength{\textwidth}{4.3in}
\raggedbottom\sloppy\sloppypar\frenchspacing

% this might be crazy, but here's my number
\setlength{\marginparsep}{0.15in}
\setlength{\marginparwidth}{2.7in}
\usepackage{marginfix} % necessary but possibly evil
\newcounter{marginnote}
\setcounter{marginnote}{0}
\renewcommand{\footnote}[1]{\refstepcounter{marginnote}\textsuperscript{\themarginnote}\marginpar{\color{darkgray}\raggedright\footnotesize\textsuperscript{\themarginnote}#1}}
\newcommand{\tfigurerule}{\rule{0pt}{1ex}\\ \rule{\marginparwidth}{0.5pt}\\ \rule{0pt}{0.25ex}}
\newcommand{\bfigurerule}{\rule{0pt}{0.25ex}\\ \rule{\marginparwidth}{0.5pt}\\ \rule{0pt}{1ex}}
\renewcommand{\caption}[1]{\parbox{\marginparwidth}{\footnotesize\textbf{\refstepcounter{figure}\figurename~\thefigure}: {#1}}}

% kill these at submission
\newcommand{\bedell}[1]{\textcolor{red}{[Bedell says: #1]}}
\newcommand{\hogg}[1]{\textcolor{magenta}{[Hogg says: #1]}}

\begin{document}

\section*{What is the sensitivity of a \\
  stellar radial-velocity observing program \\
  to an orbital companion?}

\noindent\textbf{David W. Hogg}\footnote{%
  The authors would like to thank
  [put names here]
  for help with all these concepts.
  This research was supported by the National Science Foundation and National Aeronautics and Space Administration.}\\
{\footnotesize%
  \textsl{Center for Cosmology and Particle Physics, Dept.\ Physics, New York University}\\
  \textsl{Max-Planck-Institut f\"ur Astronomie, Heidelberg}\\
  \textsl{Flatiron Institute, a division of the Simons Foundation}%
}

\medskip\noindent\textbf{Megan Bedell}\\
{\footnotesize%
  \textsl{Flatiron Institute, a division of the Simons Foundation}%
}

\paragraph{Abstract:}
We find planets and binary stars by (among other techniques) the radial-velocity
method.
Here we use information theory to estimate---or really bound---the sensitivity
of a planned or actual radial-velocity survey.
We find interesting things.

\section{Introduction}

Hello World!\footnote{Hopefully we will find a reason to cite
  \cite{wobble} but this placeholder is just so that the bibliography
  ``does something''.}
Stuff and things.

\section{The problem and solution}

\paragraph{Problem:}
Imagine that at $N$ barycentric times $t_n$ (given in $\BJD$, say) you have
measured the barycentric radial velocity $v$ of a star with uncertainty $\sigma_n$
(given in $\kms$, say). What is the sensitivity of this set of
observations to a single orbital companion with radial-velocity amplitude $\kappa$
orbiting at period $P$ and eccentricity $e$? Assume that the observer is located
isotropically relative to targets or \foreign{vice versa}, and that we
are not observing at any special time.

\paragraph{Solution:}
A single orbital companion can be parameterized by a velocity
amplitude $\kappa$ (velocity units; proportional to $M\,\sin i$), a
period $P$ (time units), an eccentricity $e$ (dimensionless), an
argument of perihelion angle $\varpi$ (in radians), and a phase $t_0$
(time). The last of these parameters could be the time at which the
companion is at conjunction or it could be the time at pericenter, or
it could be thought of as the zero of time. For any setting of these
parameters, there is a prediction $v(t_n)$ for the barycentric radial velocity of
the star at any time $t_n$ that looks like:
\begin{equation}
  v(t_n) = \kappa\,\xi(t_n - t_0; P, e, \varpi) + v_0
  \hquad,
\end{equation}
where $\xi(\Delta t; P, e, \varpi)$ is a dimensionless radial-velocity
prediction for the Kepler problem and $v_0$ is the relative radial
velocity between this star and the Solar System barycenter.

Although we don't know what priors to put on $P$ or $e$ or $\kappa$---and
we don't need to put priors on any of these---we are told that we are to
treat the angular and time distributions as isotropic.
For this reason, we can assume that we do have prior pdfs over
the argument of perihelion $\varpi$ and the reference time $t_0$:
\begin{align}
  p(\varpi) &= \uniform(\varpi\given 0, 2\pi) \\
  p(t_0) &= \uniform(t_0\given t_{\min}, t_{\min}+P)
  \hquad ,
\end{align}
where $\uniform(x\given a,b)$ is the uniform pdf for $x$ in the range $a<x<b$,
and $t_{\min}$ is some chosen reference time (like, for example, the
first time $t_1$ in the time series, but it could be anything).

\section{Discussion}

Frequentism vs Bayesianism, maybe?

% Render the references
\clearmargin\clearpage\raggedright
\bibliography{refs}

\end{document}
