% This document is part of the RvSensitivity project.
% Copyright 2020 the authors.

% TODO:

% Notes:

\PassOptionsToPackage{usenames,dvipsnames}{xcolor}
\documentclass[modern]{aastex63}

% Load common packages
\usepackage{microtype}  % ALWAYS!
\usepackage{amsmath}
\usepackage{amsfonts}
\usepackage{amssymb}
\usepackage{mathrsfs}

% Hogg's issues need to be addressed.
\renewcommand{\twocolumngrid}{\onecolumngrid} % guess what this does HAHAHA!
\setlength{\parindent}{1.1\baselineskip}
\addtolength{\topmargin}{-0.2in}
\addtolength{\textheight}{0.4in}
\sloppy\sloppypar\raggedbottom\frenchspacing % trust in Hogg

% text macros
\newcommand{\foreign}[1]{\textsl{#1}}
\newcommand{\project}[1]{\textsl{#1}}
\newcommand{\acronym}[1]{{\small{#1}}}
\newcommand{\rv}{\acronym{RV}}

\shorttitle{designing for exoplanet yield}
\shortauthors{bedell and hogg}

\begin{document}

\title{Designing an extreme-precision radial-velocity survey for exoplanet yield}

\author[0000-0001-9907-7742]{Megan Bedell}
\affiliation{Flatiron Institute, a division of the
             Simons Foundation, 162 Fifth Avenue, New York, NY 10010, USA}

\author[0000-0003-2866-9403]{David~W.~Hogg}
\affiliation{Flatiron Institute, a division of the
             Simons Foundation, 162 Fifth Avenue, New York, NY 10010, USA}
\affiliation{Max-Planck-Institut f\"ur Astronomie,
             K\"onigstuhl 17, D-69117 Heidelberg, Germany}
\affiliation{Center for Cosmology and Particle Physics,
             Department of Physics,
             New York University, 726 Broadway,
             New York, NY 10003, USA}

\begin{abstract}\noindent
% Context
New instruments are being built and new projects are being planned
that will make possible the detection of truly Earth-like exoplanets,
with masses, periods, host stars, and insolation all very like those of
Earth.
% Aims
Here we provide ideas and tools for the experimental design of such
projects, with the idea that they might be designed to maximize planet
yield, but also be usable for measuring exoplanet population
parameters.
% Methods
Our main contribution is to articulate and operationalize the
conditions that must be met for a radial-velocity signal to be adopted
or accepted as an exoplanet discovery.
These conditions are:
\textsl{(1)}~Significance of the detected radial-velocity signal,
\textsl{(2)}~specificity with respect to other orbital-companion explanations, and
\textsl{(3)}~distinguishability from non-companion stellar-variability signals.
We suggest particular implementations of these three conditions.
Our implementations are approximations to what is really done in
exoplanet-discovery contexts, but because our suggestions are
algorithmic and reproducible, they can be used in experimental-design
work.
% Results
We find that there is little difference in practice between
computationally inexpensive frequentist discovery procedures and their
more expensive Bayesian counterparts.
We find, generically, that it is useful for the observational time
baselines between radial-velocity observations to be heterogeneous;
clustered observations produce higher yields than either unclustered
or regular-grid observation plans.
% Conclusions
In the end, experimental design ought to depend on each individual
project's capabilities, objectives, and expectations; we can't solve
these problems for others. We are providing tools intended to help
diverse projects make their own choices better.
\end{abstract}

% \keywords{}

\section*{~}\clearpage
\section{Introduction} \label{sec:intro}

...Planets have been found by one h*ck of a lot of different methods!
...Comment here on some history of the development from \acronym{FFT}s
through Bayesian evidences.

...Note that this subject is important for two reasons. One is that
there really are controversies in the literature about detections, and
the language for arguing about these things could be standardized and
the considerations made clear.

...But also we need to make critical experimental design decisions,
which involve substantial financial and observing resources. These
need to be made in a framework in which we can predict exoplanet
yields under different assumptions. That is our specific motivation here.

If we had to boil the whole long literature down to basic principles,
we would see three kinds of arguments or conditions that have to be
met for a signal in an \rv\ data set to be considered strong evidence
for an exoplanet discovery:
\begin{description}
\item[Detected]
The exoplanet \rv\ signal must be significantly detected in the
data. This means, in practice, that the null (no exoplanet signal at
all) must be ruled out at good confidence, or (equivalently) that the
radial-velocity amplitude must be detected with some high fractional
precision (some number of sigma, say).
\item[Characterizable]
The exoplanet period (and perhaps other orbital parameters) must be
determined with some precision and (more importantly) not confused
with substantially longer-period, substantially shorter-period, or
(say) beat frequencies from other companions. That is, the exoplanet
signal should be uniquely determined in the sense of having orbital
parameters that are strongly preferred (statistically) to alternative
exoplanet or binary-companion explanations.
\item[Not stellar variability]
The exoplanet explanation of the \rv\ signal should be strongly
preferred to any (of some reasonable, computable set of) non-exoplanet
stellar-variability signals, for example stellar rotation,
asteroseismic modes, star spots, or stellar activity.
\end{description}
In what follows, we operationalize these and then use our
operationalized conditions to perform some toy experimental-design
studies to show how they might be used in practice.

\section{Operationalizing detection conditions}

\subsection{Detected}

Frequentist approach...

Bayesian approach will not be very different!

\subsection{Characterizable}

Frequentist approach...

Bayesian approach...

The Bayesian approach is much more expensive and involves more
questionable assumptions. Under what conditions will it be
substantially better?

\subsection{Not stellar variability}

First business: What kinds of stellar variabilities are tractable, or
convertable into a hypothesis test or parameter estimation? This list
will change as more things are discovered and worked out in the
literature.

Again, frequentist and Bayesian approaches...

\section{Example experimental design experiments}

You have five years; you can observe $M$ stars, each $N$ times. Should
you increase $N$ and reduce $M$ or \foreign{vice versa}?

...Other simple questions?

\section{Discussion}

Hello World!

\acknowledgments

It is a pleasure to thank
  Dan Foreman-Mackey (Flatiron),
  Lily Zhao (Yale),
  and
  the \project{Terra Hunting Experiment} team
for help with all the things.

\software{
    % Astropy \citep{astropy, astropy:2018},
    % exoplanet \citep{exoplanet:exoplanet},
    % gala \citep{gala},
    % IPython \citep{ipython},
    % numpy \citep{numpy},
    % pymc3 \citep{Salvatier2016},
    % schwimmbad \citep{schwimmbad:2017},
    % scipy \citep{scipy},
    % theano \citep{theano},
    % thejoker \citep{thejoker, Price-Whelan:2019a}
}

\bibliographystyle{aasjournal}
\bibliography{refs}

\end{document}
