% This document is part of the ComovingStars project.
% Copyright 2020 the authors.

% TODO:
% - Is it "comoving" or "co-moving"?

% Notes:
% - Comoving color twins?
% - Cross-match with spectroscopic surveys...

\PassOptionsToPackage{usenames,dvipsnames}{xcolor}
\documentclass[modern]{aastex63}

% Load common packages
\usepackage{microtype}  % ALWAYS!
\usepackage{amsmath}
\usepackage{amsfonts}
\usepackage{amssymb}
\usepackage{mathrsfs}

% Hogg's issues need to be addressed.
\renewcommand{\twocolumngrid}{\onecolumngrid} % guess what this does HAHAHA!
\setlength{\parindent}{1.1\baselineskip}
\addtolength{\topmargin}{-0.2in}
\addtolength{\textheight}{0.4in}
\sloppy\sloppypar\raggedbottom\frenchspacing % trust in Hogg

% text macros
\newcommand{\project}[1]{\textsl{#1}}

\shorttitle{designing for exoplanet yield}
\shortauthors{bedell and hogg}

\begin{document}

\title{Designing an extreme-precision radial-velocity survey for exoplanet yield}

\author[0000-0001-9907-7742]{Megan Bedell}
\affiliation{Flatiron Institute, a division of the
             Simons Foundation, 162 Fifth Avenue, New York, NY 10010, USA}

\author[0000-0003-2866-9403]{David~W.~Hogg}
\affiliation{Flatiron Institute, a division of the
             Simons Foundation, 162 Fifth Avenue, New York, NY 10010, USA}
\affiliation{Max-Planck-Institut f\"ur Astronomie,
             K\"onigstuhl 17, D-69117 Heidelberg, Germany}
\affiliation{Center for Cosmology and Particle Physics,
             Department of Physics,
             New York University, 726 Broadway,
             New York, NY 10003, USA}

\begin{abstract}\noindent
% Context
New instruments are being built and new projects are being planned
that will make possible the detection of truly Earth-like exoplanets,
with masses, periods, host stars, and insolation all very like those of
Earth.
% Aims
Here we provide ideas and tools for the experimental design of such
projects, with the idea that they might be designed to maximize planet
yield, but also be usable for measuring exoplanet population
parameters.
% Methods
Our main contribution is to articulate and operationalize the
conditions that must be met for a radial-velocity signal to be adopted
or accepted as an exoplanet discovery.
These conditions are:
\textsl{(1)}~Significance of the detected radial-velocity signal,
\textsl{(2)}~specificity with respect to other orbital-companion explanations, and
\textsl{(3)}~distinguishability from non-companion stellar-variability signals.
We suggest particular implementations of these three conditions.
Our implementations are approximations to what is really done in
exoplanet-discovery contexts, but because our suggestions are
algorithmic and reproducible, they can be used in experimental-design
work.
% Results
We find that there is little difference in practice between
computationally inexpensive frequentist discovery procedures and their
more expensive Bayesian counterparts.
We find, generically, that it is useful for the observational time
baselines between radial-velocity observations to be heterogeneous;
clustered observations produce higher yields than either unclustered
or regular-grid observation plans.
% Conclusions
In the end, experimental design ought to depend on each individual
project's capabilities, objectives, and expectations; we can't solve
these problems for others. We are providing tools intended to help
diverse projects make their own choices better.
\end{abstract}

% \keywords{}

\section*{~}\clearpage
\section{Introduction} \label{sec:intro}

Planets have been found by one h*ck of a lot of different methods!

If we had to boil it down, we would see three kinds of arguments or
conditions that have to be met:

1.

2.

3.

\section{Operationalizing detection conditions}

Hello World!

\section{Example experimental design experiments}

Hello World!

\section{Discussion}

Hello World!

\acknowledgments

It is a pleasure to thank
  Dan Foreman-Mackey (Flatiron),
  Lily Zhao (Yale),
  and
  the \project{Terra Hunting Experiment} team
for help with all the things.

\software{
    % Astropy \citep{astropy, astropy:2018},
    % exoplanet \citep{exoplanet:exoplanet},
    % gala \citep{gala},
    % IPython \citep{ipython},
    % numpy \citep{numpy},
    % pymc3 \citep{Salvatier2016},
    % schwimmbad \citep{schwimmbad:2017},
    % scipy \citep{scipy},
    % theano \citep{theano},
    % thejoker \citep{thejoker, Price-Whelan:2019a}
}

\bibliographystyle{aasjournal}
\bibliography{refs}

\end{document}
